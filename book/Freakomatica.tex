\documentclass[12pt]{book}
\usepackage{graphicx} % Required for inserting images
\usepackage{xcolor}
\usepackage{hyperref} %for hyper references and links
\usepackage{parskip}
\hypersetup{% properties of links
    colorlinks=true,
    urlcolor=blue,
    urlbordercolor=blue
}
\usepackage[symbol]{footmisc} %for footnote symbols


\usepackage{makeidx}
\makeindex

\title{FREAKOMATICA \\\huge{Mathematics from First Principles}}
\author{Joel Mwala}

\begin{document}
\frontmatter
\maketitle
\newpage
\begin{center}
    This page is intentionally left blank
\end{center}

\newpage



\chapter{Acknowledgements}
This section is tempolarily left blank
\newpage%!acknowledge

\chapter{Preface}
There are typical ways in which math books attempt to reach the audience.

\begin{enumerate}
      \item Dull and Solving.\\
            This is where the book only delivers the art of solving. With no care of how a audience can apply the knowledge which is being acquired. The audience is presented with the formulas neccessary to accomplish a task without caring why it is necessary to know such task and how and why was such formula was arrived at.
      \item All fun Math.\\
            This is the approach of Math facts only books. They present facts with so much assumption that you know math or with little care of whether you need to solve as you can read through the text without any use of a calculator. They are both enjoyable and exciting as they remove all the evil that is seen with math. They do a great job of showing what a beautiful land math is. All the fiction you can ever get from math land. Hence they are not used in a classroom setting where you are required to solve something so that a teacher can evaluate or for self practice.
      \item Application Math.\\
            lorem
\end{enumerate}
All These are not bad approaches they mostly achieve the purpose they are for.\\
This book tryies to blend in all these approaches mainly so that the book can be taken serious enough for use in classroom setting and so that it demystifyies math and make it fun for self learning. It presumes the reader has no prior experience with math and wants to teach from first principles. Ofcourse this is almost impossible goal. \\
The goal is to demostrate that almost anyone can contribute to this wonderful universe of math. Develop it, question it and use it in any way possible. \\
The goal is to show that seemingly so obvious ideas were not obvious before and looking into obvious things brought in advancement in human knowledge in alternative way of doing things.\\
The most notorious challege to be solved:
\begin{enumerate}
      \item people learn differently
      \item classrooms go inline with Sylabbus
      \item Back and fourth of math
      \item difficult to mix things
\end{enumerate}
\newpage%!preface

\tableofcontents
\newpage


\chapter{Introduction} %!chapter
% books overall approach
\newpage%!ch1



\mainmatter
\chapter{Things and Objects}%!chapter

\section{what are things and objects}
The world is filled with things and objects those we see and those we don't see. There are animals that are vastly available in the world of different types [examples]. Every other animal is taken to be distinct a sheep is different from another sheep. Therefore a sheep is a thing and another sheep is another thing, They are of the same type of course they belong to a group of animals called sheep. A cow is different from another cow . Therefore a cow is a thing and another cow is another thing, They are of the same type, they belong to a group of animals of cattle. This can be said of [this] thing and [another] thing. Addition property can be that "this" in [this] thing can be a sheep and "another" in [another] thing can be a cow and also this relationship of [this] and [another] is also those things are distinct. But, what if we have a [this] and another [this] belonging to one group lets say group of cattle and we have [another] thing and another [another] thing, how can we be sure that those distinct things equal to one another and how can we express it not so laboriously as we have done because we can have as many [this(es)] and [another(s)] as we want and using word [this] and [another] has shown not a good way to go about it. And Mathematics provide a way to compare a group of [this] and  and a group of [that]. mathematics provide us with numbers as a sneak way to talk about [this] and [another], a clever way to talk about quantity. This makes mathematics just another form of language in this case, a language that gives all the languages superpower to express ambiguous ideas or situations. In our case the situation was on distinctiveness how to say how much of distinct is a group from another group without comparing [this].

%one way of comparing is lining things head to head -- distinct to distinct to check if the groups are of equal or not  -- how can we express the difference of one group from another?%
\section{Glossary}
\section{References}
%!ch2


\chapter{Numbers}%!chapter
% counting and number names
% understanding place value
% reading and writing numbers
% comparing and ordering numbers
\section{Glossary}
\section{References}
\newpage


% Place Value and Operations: Introduce the concept of place value, where the position of a digit determines its value within a number. Teach addition and subtraction with regrouping (carrying and borrowing) to develop a deeper understanding of arithmetic operations.

\chapter{Counting}%!chapter
\section{Glossary}
\section{References}
\newpage

\chapter{Addition and Subtraction}%!chapter
% the numbers
% being added are called the addends, and the result the sum
% Carrying is performed when the sum of a column of addends is more than 9. Addition is
% commutative, meaning a + b = b + a
% associative, meaning (a + b) + c = a + (b + c).
% Adding zero to a number results in that same number, making zero the
% additive identity, e.g., a + 0 = a. Subtraction is the inverse of addition. In
% subtraction, for example in a − b, a is the minuend and b the subtrahend. In
% contrast to addition, subtraction is neither commutative nor associative. Just
% as carrying is often required when adding a column of numbers, borrowing is
% often required when subtracting numbers. The symbol ±, read ‘plus or minus’,
% can be used to denote an uncertainty or a pair of values (e.g., the two roots of
% the quadratic equation).
%%%%!--from 30s mathematics

\section{Glossary}
\section{References}
\newpage

\chapter{Let this be that}%!chapter
% expressions
\section{Glossary}
\section{References}
\newpage

\chapter{Multiplication and division}%!chapter
% Introduction to multiplication and division
% Multiplication tables and strategies
% Division facts and strategies
% Word problems involving multiplication and division


% Multiplication and division were extremely challenging using early
% numeral systems that did not employ positional notation, such as Egyptian,
% Greek, and Roman numerals. The numeral and arithmetic system eventually
% adopted in Europe was developed in India, with important advances made in
% the sixth and seventh centuries. In the multiplication a × b = c, a is the
% multiplier, b the multiplicand and c the product; a and b are also called
% factors. Notation for multiplication of two numbers a and b includes a × b, a ·
% b, (a)(b) and, favoured by mathematicians, simply ab. Similar to addition,
% carrying is necessary when the product of a column of digits is more than 9.
% In the example a × 1 = a, 1 is the multiplicative identity. Multiplication is
% commutative, meaning a × b = b × a, and associative, meaning (a × b) × c = a ×
% (b × c). Division is neither. In the division a ÷ b = c, a is the dividend, b the
% divisor, and c the quotient. Mathematicians favour the notation a/b to a ÷ b.
% Long division is a division algorithm that displays the dividend (the amount to
% divide), divisor (number you divide by), and quotient (the answer) in a tableau.
% For mathematicians, division of any number by zero is undefined because it
% doesn’t make sense in a rigorous manner.


% Multiplication is repeated addition of a first number a specified second
% number of times. Division is determining how many times one quantity is
% contained in another


%%%%!--from 30s mathematics


\section{Glossary}
\section{References}
\newpage

\chapter{Fractions, decimals and Percentages}%!chapter
% Introduction to fractions
% Parts of a fraction (numerator, denominator)
% Comparing and ordering fractions
% Adding and subtracting fractions with common denominators
% Introduction to decimals
% Reading and writing decimals
% Decimal place value
% Converting decimals to fractions and percentages
\section{Glossary}
\section{References}
\newpage

% properties of numbers
% commutative ....
% associative

\chapter{Order of Operations}%!chapter
\section{Glossary}
\section{References}
\newpage

\chapter{Operations all together}%!chapter
\section{Glossary}
\section{References}
\newpage

\chapter{Power}%!chapter
\section{Glossary}
\section{References}
\newpage

\chapter{Root}%!chapter
\section{Glossary}
\section{References}
\newpage

\chapter{Where to go from here}
% Guidance for further math learning and resources
% Summary of the book's content and achievements
\newpage

\backmatter
\printindex
\end{document}


Attempts at solutions had wonderful unforeseen
benefits

##links to youtube channels
## links to specific video

## applications of the gist
- physics,
- technology, biology, medicine, geology, psychology, linguistics, politics, law,
- theology, history, philosophy and mathematics

## its all symbols
##Historical figures
- galileo
- fibonacci
- zeno
- GOTTFRIED LEIBNIZ
- Alkwazir

##contributions
- contributed by everyone no matter the occupation
---words
//multiplier
//multiplicand
//devisor
//numerator
//dividend
//idea of zero
//variables
//coefficient
//idea of infinity
//zeno's paradox
// reversal reversal
//"poor language condemns us to poor thought"
##Incompleteness
//parallel lines
//absolute values
%table of contents

1 Acknowledgements
2 Introduction
3 Why Mathematics
4 Branches of Mathematics
6 Numbers
7 Counting
8 Addition and Subtraction
9 Let this be that
10 Multiplication and division
11 Fractions, decimals and Percentages
12 Order of Operations
13 Operations all together
14 Power
15 Root
16 Where to go from here


%%!NO PLACE IN LONDON
# Algebraic Concepts
- Introduction to variables and expressions
- Solving equations and inequalities
- Evaluating expressions
- Introduction to functions
- Geometry

# Basic geometric
- shapes (lines, angles, triangles, quadrilaterals, circles)
- Perimeter and area
- Volume and surface area
- Introduction to coordinate geometry

#logarithms
#function
#Sequencies
- fibonacci sequence

#combinations
#solids
#Trigonometry
#probability?

#goal
real-world applications
interactive
fostering understanding
curiosity
solve every day problems
simulations
put into game


BOOKS
30 second mathematics
How to solve it - George Poyla
Mathematician's Appology - G.H. Hardy